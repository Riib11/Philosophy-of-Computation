
\documentclass{article}
\usepackage[margin=1.4in]{geometry}
\usepackage{color}
\usepackage{caption}
\usepackage{hyperref}
\usepackage{csquotes}
\usepackage{amsmath}
\usepackage{amssymb}
\usepackage{soul}
\usepackage{changepage}
\usepackage{alg}
\usepackage{graphicx}
\graphicspath{ {./} }
\usepackage{listings}
\lstset{aboveskip=3mm, belowskip=3mm, showstringspaces=false, columns=flexible, basicstyle={\small\ttfamily}, numbers=none, breaklines=true, breakatwhitespace=true, tabsize=3}

\usepackage{newlfont}
\usepackage{program}
\catcode`\_\active

\newcommand{\define}[1]{{\sc Definition.} \textbf{#1}: }

\begin{document}
\begin{center}{\huge   Benchmarking Google Translate }\\[0.4cm]{\large  Philosophy of Computation Lab IV }\\[0.75cm]{\large  Henry Blanchette }\\[0.5cm]{\large  April 5, 2019 }\\[1.0cm]\begin{abstract}
    TODO
\end{abstract}\end{center} \section{Introduction}\section{Transcript Setup}


I selected transcript sections that reflect a variety of writing styles, including modern English, English, technical writing, storytelling, and English translated from other languages.




\vspace{1em} \noindent
Transcripts:
\begin{enumerate}
  \item[T1.] The Bible, Genesis
  \item[T2.] Bedau's patentsample.txt
  \item[T3.] Shakespeare's Henry IV, Part 1
  \item[T4.] Melville's Moby Dick, Chapter 1
  \item[T5.] Mariam-Webster English Dictionary, definition of Abdicate
\end{enumerate}

\section{Translation task}


Start with a sequence of languages $L_0, \dots, L_n$ and a transcript in $L_0$, called the \textit{original transcript.}
GT translates the $L_0$-transcript to $L_1$, and then translates the resulting $L_1$-transcript to $L_2$, and so on until the transcript has been translated to $L_n$.
At the end, there is left a $L_n$-transcript.
Then, GT translates this $L_n$-transcript back to $L_0$ - the result is the \textit{processes transcript}.
The differences between the original and processed transcripts are measured to rate GT's success at this task.
The goal of this scoring is to rate GT according to how well it preserve the meaning and grammatic structure of the original transcript.


\section{Translation Success Measure}


I rate GT's success at the task by how close the processed transcript is the the original transcript in terms of meaning and grammar. For each of these dimensions, I categorized an ordered ranking system.




\textbf{Grammar} is how properly-constructed the processed transcript is according to the rules of $L_0$ and the grammatical structure of the original transcript.
The following are the grammar classes I used in order of increasing success.
\begin{enumerate}
  \item[G1.] completely confused
  \item[G2.] mostly confused
  \item[G3.] often confused
  \item[G4.] sparsely confused
  \item[G5.] passing
  \item[G5.] perfect
\end{enumerate}




\textbf{Meaning} is how close the processed transcript is to the original transcript in meaning.
The following are the grammar classes I used in order of increasing success.
\begin{enumerate}
  \item[M1.] no meaning
  \item[M2.] irrelevant
  \item[M3.] sparsely relevant
  \item[M4.] often relevant
  \item[M5.] mostly accurate
  \item[M6.] perfect
\end{enumerate}


\section{Experiment 1: Well-Documented Language Translation Ring}\subsection{Language Setup}


I selected from the top 5 languages (without English) by native speaker count . I hypothesized that this would correlate with the amount of effort that Google has put into training translations to and from these languages, which should yield more coherent and thus easier-to-score processed texts from this task.




\vspace{1em} \noindent
The following are the languages used in this experiment in order of decreasing native speakers count:
\begin{enumerate}
  \item[L1.] Chinese (simplified)
  \item[L2.] Spanish
  \item[L3.] Hindi
  \item[L4.] Arabic
  \item[L5.] Portuguese
\end{enumerate}

\subsection{Experimental Design}


I ran each transcript through the following trials, where the selected languages and their order was chose randomly:
\begin{enumerate}
  \item[] Trial 1: Chinese $\rightarrow$ Arabic $\rightarrow$ Spanish $\rightarrow$ Portuguese $\rightarrow$ Hindi
  \item[] Trial 2: Hindi $\rightarrow$ Chinese $\rightarrow$ Portuguese $\rightarrow$ Arabic $\rightarrow$ Spanish
  \item[] Trial 3: Hindi $\rightarrow$ Spanish $\rightarrow$ Arabic $\rightarrow$ Chinese $\rightarrow$ Portuguese
  \item[] Trial 4: Chinese $\rightarrow$ Arabic $\rightarrow$ Spanish $\rightarrow$ Portuguese $\rightarrow$ Hindi
  \item[] Trial 5: Arabic $\rightarrow$ Chinese $\rightarrow$ Portuguese $\rightarrow$ Spanish $\rightarrow$ Hindi
\end{enumerate}


\subsection{Predictions}\subsection{Results}\subsection{Analysis}\section{Experiment 2: Under-Documented Language Translation Ring}\subsection{Language Setup}


I selected from the bottom 5 languages by native speakers that Google Translate supports .




\vspace{1em} \noindent
Languages:
\begin{enumerate}
  \item[L1.] Nepali
  \item[L2.] Sinhala
  \item[L3.] Greek
  \item[L4.] Hungarian
  \item[L5.] Zulu
\end{enumerate}

\subsection{Experimental Design}


I ran each transcript through the following trials, where the selected languages and their order was chose randomly:




\begin{enumerate}
  \item[] Trial 1: Zulu $\rightarrow$ Hungarian $\rightarrow$ Nepali $\rightarrow$ Sinhala $\rightarrow$ Greek
  \item[] Trial 2: Nepali $\rightarrow$ Hungarian $\rightarrow$ Greek $\rightarrow$ Sinhala $\rightarrow$ Zulu
  \item[] Trial 3: Nepali $\rightarrow$ Sinhala $\rightarrow$ Zulu $\rightarrow$ Greek $\rightarrow$ Hungarian
  \item[] Trial 4: Sinhala $\rightarrow$ Greek $\rightarrow$ Nepali $\rightarrow$ Zulu $\rightarrow$ Hungarian
  \item[] Trial 5: Hungarian $\rightarrow$ Greek $\rightarrow$ Sinhala $\rightarrow$ Zulu $\rightarrow$ Nepali
\end{enumerate}

\subsection{Predictions}\subsection{Results}\subsection{Analysis}\section{Conclusion}



\section*{Bibliography}
\end{document}